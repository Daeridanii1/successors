\documentclass[11pt]{report}
\title{% 
    The Successors \\ 
    \smallskip
    \large A Ten Thousand Year Fictional History}
\author{``Daeridanii''}
\date{Last Edited 13 August, 2023}
\begin{document}
    \maketitle
    \tableofcontents
    \newpage
    \section{Introduction}
    The Successors are a alien species intended for the open source game Endless Sky. This document is patterned off of other ones compiled for other species in an effort to deepen their backstory and narratively justify their presence.
    
    This document is intended to lay out the various lore and worldbuilding elements the Successors have been constructed with. It is not a comprehensive document, nor is it intended as immutable---but the hope is that laid out in the following chapters is a detailed and interesting enough description of the species to meaningfully inform the stories that are told with and about them.

    Finally, as of the time of writing, this document is a work-in-progress! Several sections are incomplete.

    \section{Biology}
    I have affectionally referred to the Successors as ``frogtopi'' in that the basis from their design springs from a combination of amphibian and cephalopod traits. The basic body plan is that of a invertebrate body with a ``head'' at one end with two eyes and eight muscular arms at the other, which are used for movement and object manipulation. Adults stand modestly taller than humans; between two and three meters, with significant variation between individuals. 

    Successor skin is moist and mucosal, and like octopi, they are capable of changing the color and texture of their skin. This ability can be used for camouflage, but in modern Successor society, is primarily a communicative method which they use in tandem with vocalized language to express themselves. With regard to how they are written, one can tell what mood a Successor is in by what color they are---an important note.
    
    Successors are amphibious\footnote[1]{It should be noted that locomotion in a non-aqueous environment is probably going to be difficult for an invertebrate without an exoskeleton. As Endless Sky is not, to my knowledge, a biologically-rigorous game, I think making them invertebrates is a passable conceit, but I have been exploring the plausibility of incorporating semi-rigid non-bone elements into their speculative biology. For example, several primitive fish like hagfish, lampreys, and coelecanths retain notochords past the embryonic stage---these structures provide stiffness like a spine might while not \emph{technically} constituting a spine or necessitating an organism's classification as vertebrate.}; like frogs, they breathe through their skin and are comfortable both underwater and on land. While on land, they are most comfortable in a moist environment, and becoming dried-out can be dangerous or lethal to them. Architecture is designed around this: the interiors of buildings are humid, floors are covered in thin layers of water, etc. The Successors have a significant underwater habitation presence on the planets they occupy, which the player is not as readily exposed to as their dry-land facilities. They are comfortable in colder temperatures than humans: environmentally-conditioned areas will probably be in the range of 0-10 Celsius (32-50 Fahrenheit).
    
    Successors are a hermaphroditic species; adult individuals do not differ in their sexual capabilities, and they have no distinct sex. In a societal sense, this means there are no distinct gender roles or norms; from a narrative standpoint, this means all Successors are referred to in gender-neutral terms.
    \section{History}
    \subsection{Early History (more than 10\,000 years before present)}
    The Successor species originates from the planet presently known as Kella-Uuoru-Sossa (\emph{lit. ``first home-place''}), a habitable, watery planet orbiting a harsh A-type star in the galactic arm. Kella-Uuoru-Sossa is slightly smaller and substantially less massive than Earth, with a surface gravity of about 0.5 g and a thick, oxygen-rich atmosphere. It orbits at substantial distance from its parent star, leading to very long years of approximately 1600 Earth days and generally colder surface temperatures than on Earth. It has two moons: the nearer is smaller and irregular and the more distant is larger with a thin atmosphere. Both Kella-Uuoru-Sossa and its moons are rich in minerals that fueled technological development from early in Successor history up until the high point of their interstellar society.
    
    For much of recorded Successor history, settlements were restricted to water-adjacent areas as Successor biology requires constant moisture; many inland settlements were only made after the development of weather control technology in the mid-spaceflight era, at which point coastal regions were mostly filled with large cities.

    The tendency towards aristocracy in Successor society can be traced back to this time as well. During this era, most of the society was separated into long, thin strips along rivers and lakes ruled over by local noble families. With technological advancement came greater connection between these groups, but the unfeasibility of inland expansion and presence of many geograpical divisions kept them politically disparate. Many of the modern High Houses (see Society:Politics) were nominally formed before the spaceflight era and have persisted to the present day. In reality, there is no evidence to suggest than any of the modern Houses have such a long or unbroken history, but instead have appropriated or rewritten the names of these ancient feudal powers in an effort to grant their modern claims to power a greater sense of authenticity. 

    \subsection{Spaceflight (10\,000 -- 5000 YBP)}

    Early spaceflight began approximately 10\,000 years before the present with primitive chemical rocketry, much like humanity. From this point, technological progress was initially slow: spaceflight was simply not economically or politically valuable at this stage in Successor history, with many of the Successors' efforts put towards the exploration and colonization of inland portions of their planet rather than outer space. This led to the development of sophisticated environmental control and terraforming systems and significant geographical changes to much of the landmass of Kella-Uuoru-Sossa. Spaceflight remained a low-level priority for nearly three hundred years after its invention, only rising to administrative and popular attention in the wake of war.

    What started as a series of relatively routine disputes over inland irrigation eventually became what would come to be called the Ash War. With many of the now-established inland governments wholly reliant on artificial irrigation to keep their territories habitable, the potential destruction of these irrigation systems was a salient and pressing threat. Those that were exposed were quickly destroyed to force surrender, but those that were well-fortified necessitated a longer-term approach: thousands of square kilometers of land were burned in order to poison the air and rain. It lead to the deaths of about 2\% of the planet's population and many years of environmental damage that reduced quality of life worldwide.
    
    Space had suddenly became a direction for escape from the planet's turmoil, and in the following years, many of the remaining planetary governments focused substantial effort on spaceflight, culminating in settlements being established on the larger of the two moons of Kella-Uuoru-Sossa and eventually on other planets in the system.

    The development of the hyperdrive came after another three hundred years, by which point the Successor species had established itself as a sustained multiplanetary group within their star system. Mining for the raw materials used in hyperdrives suddenly became an especially lucrative endeavour, and the industrial might of the disparate governments blossomed as they quickly developed early interstellar vessels.

    \subsection{The Golden Age (5000 -- 1800 YBP)}
    \subsection{Cataclysm \& Collapse (1800 -- 1200 YBP)}
    Cataclysm emerged slowly but with a vengeance. 

    The Predecessors, envious of the Quarg, and inspired by the wormhole, finally managed to create a working equivalent of the jump drive: the Point Drive was complete, and it allowed the Predecessors to explore beyond the limits imposed on them by the Archons. The point drives developed by the Predecessors were not a clean or particularly precise technology, however. Like with the Korath, production of these drives was a dangerous and intensive process. By separating the warp and weft\footnote[2]{For more information on the weft, please read the Endless Sky history documents written by MZ or Ravenshining, which discuss it further.} elements of the drive inside the passage of the wormhole, the drive was left ``split'' and capable of enabling travel through weft-space. Splitting the drive in this fashion, however, also left it embedded in the local fabric of space, and so every time such a drive was used, it would pull and tear on space as it moved its target. This effect, while small for each individual ship and jump, accumulated in Predecessor core systems as the manufacturing and consequent use of new point drives continued to grow.

    The first noticed effects were the weakening of the hyperspace lanes. Freighters began to ``fall out'' of transit midway between systems; some disappeared entirely. On occasion, the enormous energy release of a hyperdrive's activation simply had nowhere to go, and ships exploded preparing to depart for seemingly no reason. The only way to avoid this was, of course, to use a point drive instead, which continued to work as usual. The high cost of this equipment, however, precluded this solution for many Predecessors. Consequently, there was an exodus to the parts of Predecessor territory which remained unaffected: the cluster through the wormhole. This was the start of the collapse.

    The disruption and destruction of local space continued, however, and began manifesting in increasingly-tangible forms. Spatial rifts emerged in distant orbits, swallowing light and tearing apart anything which got close. On the homeworld, population decline and reduced agricultural yield began to drive an economic crash. 

    \subsection{Sustained Existence (1200 -- 500 YBP)}
    \subsection{Regrowth (500 YBP to present)}
    \section{Society}
    \subsection{Naming}
    Successor names are tripartite.
    
    An individual's first name is their family name, representing the direct lineage of which they are a part. Children generally inherit the more prestigious of their parents' family names, but in some circumstances, such as of equally prestigious family lines, the parents' names may be combined or blended and passed on to the children in that format.

    The second of an individual's names is their personal name, generally chosen by an individual's parents. Like with the family name, personal names may be inherited from prestigious members of the family, or simply given for other reasons.

    The third name is the chosen name, generally chosen by the individual upon entering adulthood as a representation of their interests, accomplishments, or connections. This is the name by which the majority of Successors will introduce themselves.

    \subsection{Politics}
    Successor society is highly family-centric. In general, Successors have large, tightly-knit families, and one's belongingness to one's family is considered the most important belongingness of all. 

    The Successor political structure is an extension of this. I would characterize it as a ``somewhat representative aristocracy.'' The largest political units are the High Houses: the noble families of the Successors and their associated supporters. Nominally, inclusion in the ``supporters'' part of this includes some degree of blood relation, but practically, in the modern day the bar for inclusion is lower, as the Houses vying for power want the largest group of supporters possible. That being said, this is still along the lines of ``family,'' and so children of those belonging to one of the High Houses become members themselves by default. As a consequence, many Successors are members of multiple of the High Houses, even if those Houses might be politically or socially in conflict. This is part of the reason for the unique political situation present in Successor society by the time of the player's arrival; the High Houses are highly disincentivized against open warfare against each other because their popular support is shared with their competitors.
    
    The High Houses are still highly aristocratic despite this popular element. Those in control, generally referred to as Scions, are those belonging to the core supposedly-unbroken bloodline of the House; a single House may have around a dozen---parents, children, siblings, etc, who belong to the current generation of the core of the family. In practice, this is a little more complex: the Houses are supported by complex bureaucracies that wield significant power and influence. These consist of advisors, representatives, trusted military members, and more distant relatives. That being said, it is almost always the case that the Scions wield more power than mere figureheads, even if their power is generally not absolute.
    
    The High Houses \emph{do} have a popular element, though. While the Scions and bureaucracy hold power within the House, it is the support and taxation of their general members that allow them to wield and project external power. By the start of the player's involvement in their story, the remaining High Houses are divided into two loose alliances: the Old Houses and the New Houses.
    
    \subsection{Old and New Houses}
    The High Houses in the star cluster the main part of the story takes place in are grouped in the Old Houses in the north, and the New Houses in the south. The history behind this division goes as such: the modern Successor territory is a small part of their historical empire, which collapsed ca. 2000 years prior. During that collapse, the High Houses that relocated to the area as refugees were declared the ``New Houses,'' and those which were already established in the territory were the ``Old Houses.'' This is complicated further because, in general, the New Houses are older than the Old Houses; they hail from the central parts of the empire and were therefore historically more powerful and established than the Old Houses, which were comparatively less prestigious and relegated to the ege of the empire.

    By the modern day, the Old Houses are generally the group ``in control''; when the two groups come into conflict, the general situation is that the Old Houses are the government and the New Houses are the opposition. The New Houses control a smaller and less comfortable territory with fewer natural resources, a smaller population, and on average a slightly poorer citizenry and aristocracy. It's worth noting that this is not a large or sharp division; the New Houses are societally disadvantaged, but ultimately not by much, and they generally make up for their slightly lower wealth and manufacturing volume by being more efficient and intentional with their governance.

    Each House controls its own planet, and each of the groups (Old and New) control a collective planet as part of their alliance, where no individual House has specific authority.


    \subsubsection{The Old Houses}

    \noindent
    \emph{House Aqrabe}

    Of all the Old Houses, House Aqrabe is the most neutral. They are seated at the temperate moon of Mavra-Sol-Kvel, and have a strong commercial economy built on trade flowing through their system, involving both the Old and New Houses as well as the general Successors. Because of their relative neutrality, they often work together with House Kaatrij in a sort of peacemaking coalition.
    \bigskip

    \noindent
    \emph{House Sioeora}

    The smallest of the Old Houses, House Sioeora controls the far northern planet of Staja-Kella-Oa. They are strongly connected to the educational institutions of Staja-Kella-Oa, which take a particular interest in history and the sciences and are generally regarded as the best in Successor space. While Sioeora cannot compare with the economic might of the other Old Houses, they serve as the cultural core of the group, preserving Successor history and spearheading the Old Houses' cultural and scientific development efforts, which grants them significant influence.

    Staja-Kella-Oa once was home to a sapient race circa 100\,000 years prior to the events of the story, at which time they went extinct.

    Sioeora lend the player the translation device that allow them to communicate with the Successors.
    \bigskip

    \noindent
    \emph{House Chydiyi}

    Chydiyi is the richest of the Old Houses, with an economy based primarily on their position as the strongest manufacturing powerhouse in Successor space. Their seat of power is the material-rich Khora-Vasa-Reyyaa, where they have long-established mining and manufacturing projects. They are also the most militarized of the all the High Houses, contributing a majority of the Old Houses' collective fleet in exchange for a significant financial inflow from their allies.

    Because of their significant military and economic power over the other High Houses, Chydiyi is considered the \emph{de facto} leader of the four.

    Because Khora-Vasa-Reyyaa is the most material-rich planet in Successor space, House Chydiyi is bound by several treaties that guarantee its continued raw material exports to the other High Houses, both Old and New.
    \bigskip

    \noindent
    \emph{House Myurej}

    Myurej represents the largest population of any of the High Houses in Successor space, concentrated on its familial home of Raaqa-Kvelq-Ryuit. Like Aqrabe, it enjoys an economy built on the taxation of trade passing through its system and of the diverse incomes of its members. It is also the oldest of the Old Houses. 
    \bigskip

    \subsubsection{The Old Houses}

    \noindent
    \emph{House Seineq}

    Seated on the planet of Yoqqa-Vasa-Vasa, House Seineq is the wealthiest and most powerful of the New Houses. They are also most proximate to the New Houses' council moon of O-Vasa-Oa (as it orbits their home planet). Seineq also operates the majority of the New Houses' shipyards.
    \bigskip

    \noindent
    \emph{House Haeora}

    House Haeora represents the largest population among the New Houses, with a strong services-based economy centered on its home planet of Moliia-Cosa-Vasa.
    \bigskip

    \noindent
    \emph{House Kvasq}

    Kvasq mirrors Chydiyi with an economy based on extraction and manufacturing from its seat of power on Kasii-Cavasaa-Oa. They are highly self-sufficient; while they are still connected to the other New Houses politically and economically, and generally a predictable member of that alliance, they do so more out of expedience and utility than any sense of unity.
    \bigskip

    \noindent
    \emph{House Kaatrij}

    Seated on the rainy planet of Raaqa-Puan-Uuoru, House Kaatrij has perhaps the most interesting story of any of the High Houses. Unlike the rest, they do not claim an unbroken lineage since their founding, but instead recognize that during the collapse, all the Scions eventually died without direct heirs. To continue the House, many of the refugees from the collapsing Houses that did not escape the bedlam of the central empire, were brought into the fold and inducted as full members of the house and its family.

    In the present day, Kaatrij remains the smallest of all the High Houses, and it has little direct influence outside its home system. However, it has earned a reputation as efficient and adaptable, and serves as a center of education and high-tech research and development in the south of Successor space.

    Because of its more democratic and egalitarian history, it incorporates a more diverse leadership and tends to be the most neutral of the New Houses in general conflicts with the Old.
    \bigskip

    \subsection{Language}
    \subsection{Art \& Performance}
    \subsection{Cuisine}
    \section{Technology}
    Successor technology is primarily based around their advanced materials science; this is reflected in their defensive focus of strong and adaptable hull armor over shielding. This armor is not purely an inorganic composite, but is synthesized over scaffolds by bioengineered bacteria which themselves are crystallized and immured in the material as it forms.

    The Successors use almost exclusively kinetic weapons---that is, that their weapons impart their destructive power through kinetic energy. This is partially based in a theory of aesthetics proposing that these weapons are more ``elegant'' or ``beautiful'' than other options.

    \subsection{Modern}
    \subsection{Heirlooms}
    \section{Ending Notes}
    The lore documents written by Michael Zahniser and Lia Gerty ``ravenshining'' (``a brief history of the galaxy'' and ``On the Korath,'' respectively) were of great help in compiling this document.

\end{document}